\AtBeginDocument[seminar]{%
  \pdfhorigin=1sp
  \pdfvorigin=1sp
  \paperwidth=297truemm
  \paperheight=210truemm
}

\documentclass[article]{seminar}

\usepackage{multicol}
\usepackage{tabularx}

\usepackage{nth}

\usepackage{color}
\definecolor{MyDarkBlue}{RGB}{71,57,146} % BlueViolet
\definecolor{MyDarkRed}{RGB}{139,00,00}

\usepackage[bookmarks,breaklinks=true,%
	colorlinks=true,linkcolor=MyDarkBlue,urlcolor=MyDarkRed,%
	pdftex,%
	pdfauthor={Vladimir Kotal, Jan Pechanec},%
	pdftitle={Unix programming in C},%
	pdfsubject={Materials for the NSWI015 class at MFF-UK},%
	pdfkeywords={Unix, programming, C, MFF, MFF-UK},%
	pagebackref=true,%
	]{hyperref}

\usepackage{url}
% by default \url will use monospaced font. suppress this and use normal font.
\urlstyle{same}

% for strike-through (\sout)
\usepackage[normalem]{ulem}

\usepackage{alltt}

\usepackage{upquote}

%%% Macros %%%

% (janp)
\chardef\clqq=254  \sfcode254=0 \lccode254=0
\chardef\crqq=255  \sfcode255=0 \lccode255=0
%\DeclareRobustCommand\uv[1]{{\leavevmode\clqq#1\crqq}}
\DeclareRobustCommand\uv[1]{{\leavevmode{},,#1''}}
% (janp) end

\renewcommand{\slidelabel}{}
\setlength{\textwidth}{0.9\textwidth}

\newcommand{\sltitle}[1]{{\centering\textbf{\Large #1}
    \vskip 2em plus 0pt minus 2em\par}} % Slide title

% bold
\newcommand{\emsl}[1]{\textbf{#1}} % Emphasizing in slides

\newcommand{\emprg}[1]{\emph{\color[rgb]{1,0,0} #1}} % Emphasize in programs
\newcommand{\emblue}[1]{\emph{\color[rgb]{0,0,1} #1}} % emph in blue

% my very important note
\newcommand{\rednote}[1]{\color[rgb]{1,0,0} #1}

% What should be checked
\newcommand{\CHECK}[1]{{\color[rgb]{1,0,0} $\star$#1$\star$}}

\newsavebox{\boxTMP}
\newcommand{\raisetab}[1]{ % Align first table row with other text
    \sbox{\boxTMP}{\begin{tabular}{c}\hline X\\\hline\end{tabular}}
    \raisebox{\ht\boxTMP}{#1}}

\newcommand{\funnm}[1] {% Emphasized function name
    {\bf #1}}

\newcommand{\funml}[1] { % Multi-line function prototype
    \begin{minipage}{\slidewidth}
    \vspace{-1ex}\texttt{\begin{tabbing}#1\end{tabbing}}
    \end{minipage}}

\newcommand{\bs}{\char92\relax} % TT backslash
% http://tex.stackexchange.com/questions/9363/how-does-one-insert-a-backslash-or-a-tilde-into-latex
\newcommand{\tilda}{\char`\~} % TT tilda

\newtoks\prgcharsI\newtoks\prgcharsII
{\catcode`\_=13\catcode`\&=13\global\prgcharsI={_}\global\prgcharsII={&}}
\def\prgchars{ % Do not require backslashes for these characters often used
                % in C program source code
    \catcode`\_=13\catcode`\&=13
    \expandafter\def\the\prgcharsI{\_}\expandafter\def\the\prgcharsII{\&}}

\centerslidesfalse

\newcounter{topicval}

\newcommand{\slidecontents}[1]{%
  \setcounter{topicval}{#1}%
  \begin{itemize}
  \topic{1}{Introduction, Unix and C, programming tools}
  \topic{2}{Basic Unix concepts and conventions, its API}
  \topic{3}{Access rights, devices}
  \topic{4}{Process manipulation, program execution}
  \topic{5}{Signals}
  \topic{6}{Process synchronization and interprocess communication}
  \topic{7}{Network programming}
  \topic{8}{Programming with threads}
  \topic{9}{Appendix}
  \end{itemize}%
}
\newcommand{\topic}[2]{%
  \item\relax
  \ifnum #1=\value{topicval}%
    \textbf{#2}%
  \else
    #2%
  \fi
}

\begin{document}

\begin{slide}
\centerslidestrue
\begin{center}
\title{\LARGE Unix/Linux Programming in C}
\author{(NSWI015)}
\date{Version: \rm\today}
\maketitle

\vspace{2ex}
{\small (c) 2011 -- 2023 Vladim\'{i}r Kotal}\\
{\small (c) 2005 -- 2011, 2016 -- 2022 Jan Pechanec}\\
{\small (c) 1999 -- 2004 Martin Beran}

\vspace{2ex}
Department of SISAL\\
Faculty of Mathematics and Physics, Charles University\\
Malostransk\'{e} n\'{a}m. 25, 118 00 Praha 1

\end{center}
\end{slide}

\begin{itemize}
\item This is official material for the class \emph{Unix/Linux Programming in C}
(NSWI015) lectured at the Faculty of Mathematics and Physics, Charles University
in Prague.
\item This material is published under the
\href{http://creativecommons.org/licenses/by-nc-sa/3.0/cz/}{Creative Commons
BY-NC-SA 3.0} license and is always a work in progress, see the history on
GitHub:\\
\url{https://github.com/devnull-cz/unix-linux-prog-in-c}
\item To download the latest version, go to the
\href{https://github.com/devnull-cz/unix-linux-prog-in-c/releases}{releases}
on GitHub.
\item Source code referenced from this material is published in
\href{http://creativecommons.org/licenses/publicdomain/}{Public Domain} unless
specified otherwise in the files.
\item The source code files can be found on GitHub here:\\
\url{https://github.com/devnull-cz/unix-linux-prog-in-c-src}
\item In case you find any errors either in the text or in the example programs,
we would appreciate you letting us know. Especially do not hesitate to create new
issues on \url{https://github.com/devnull-cz/unix-linux-prog-in-c/issues}.
\end{itemize}

\pagebreak

\begin{slide}
\sltitle{Contents}
\slidecontents{0}
\end{slide}

\begin{itemize}
\item This lecture is mostly about Unix principles and Unix programming in the~C
language.
\item \emsl{The lecture is mostly about system calls, i.e. an interface between a
user space and system kernel.}
\item For the API, we will follow the \emph{Single UNIX Specification,
version~4} (SUSv4). Systems that submit to the Open Group for certification and
pass conformance tests are termed to be compliant with the UNIX standard
UNIX~V7.  Some versions of AIX, HP-UX and macOS on selected architectures
are compliant with the previous version SUSv3
(\url{http://www.opengroup.org/openbrand/register/xy.htm}).
\item The specific source code examples linked from this material are usually
tested on Solaris, macOS and Linux.
\end{itemize}

%%%%%
\pdfbookmark[0]{intro, programming utilities}{intro}

\begin{slide}
\sltitle{Contents}
\slidecontents{1}
\end{slide}

\pdfbookmark[1]{Current UNIX and Unix-like Systems}{currentunix}

\begin{slide}
\sltitle{Proprietary UNIX and Unix-like Systems}

\begin{itemize}
\item Sun Microsystems, now Oracle: \emsl{SunOS} (defunct), \emsl{Solaris}
\item Apple: \emsl{macOS} (formerly Mac OS X, Mac OS)
\item SGI: \emsl{IRIX} (in maintenance mode)
\item IBM: \emsl{AIX}
\item HP: \emsl{HP-UX}, \emsl{Tru64 UNIX} (defunct, formerly by Compaq)
\item SCO: \emsl{SCO Unix} (discontinued)
\item BSD/OS: \emsl{BSDi} (discontinued)
\item Xinuos (formerly Novell): \emsl{UNIXware}
\end{itemize}
\end{slide}

\begin{slide}
\sltitle{Open source Unix-like Systems}

\begin{itemize}
\item rather extensive number of \emsl{Linux} distributions
\item \emsl{FreeBSD}
\item \emsl{NetBSD}
\item \emsl{OpenBSD}
\item \emsl{DragonflyBSD}
\begin{itemize}
\item all BSD variants have roots in the 4.3BSD-Lite source code
\end{itemize}
\item \emsl{Minix}, micro-kernel based
\item \emsl{Illumos}, based on Solaris
\end{itemize}
\end{slide}

\begin{itemize}
\item Note that \emsl{Linux is a kernel}, not the whole system.  In contrast to
FreeBSD for example, which covers both the kernel and the userland.  It is
better to say a ``Linux distribution'' if you discuss a whole system that is
built around the Linux kernel.
\item FreeBSD and NetBSD forked from 386BSD (now defunct) in 1993, OpenBSD
forked from NetBSD in 1995, and DragonflyBSD forked from FreeBSD in 2003.
386BSD itself was based on 4.3BSD-Lite.  However, the history is much more
complicated, as usual.
\item Presently, the ``UNIX'' trademark can be only used by systems that passed
conformance tests defined in the Single UNIX Specification (SUS).
\item From those systems listed above, only macOS, AIX, and HP-UX are
UNIX~03 compliant (\url{http://www.opengroup.org/openbrand/register/}).  Other
non-certified systems, are often described as ``Unix-like'', even when in many
cases they closely follow the standard.  However, the word ``Unix'' is often used
for systems from either group.
\item The above list is a tiny fraction of the whole Unix world.  Every
proprietary Unix variant likely came from either UNIX~V or BSD, and added its
own features.  This resulted in quite a few standards as well, see page
\pageref{UNIXSTANDARDS}.  In the end vendors agreed upon a small set of those.
\item If you are interested in a detailed and up-to-date Unix system version
history, go check \url{https://www.levenez.com/unix/}.
\end{itemize}

%%%%%

\pdfbookmark[1]{UNIX standards}{unixstd}

\begin{slide}
\sltitle{UNIX standards}
\begin{itemize}
\renewcommand{\baselinestretch}{0.8}
\item \emsl{SVID} (System~V Interface Definition)
\item \emsl{POSIX} (Portable Operating System based on UNIX)
\item \emsl{XPG} (X/Open Portability Guide)
\item \emsl{Single UNIX Specification}
\end{itemize}
\end{slide}

\label{UNIXSTANDARDS}

\begin{itemize}
\item The very basic information is that the area of UNIX standards is very
complex and incomprehensible on a first sight.
\item AT\&T allowed the producers to call its own commercial UNIX variant
``System V'' only if it complied to the SVID standard conditions. AT\&T also
published \emph{System~V Verification Suite} (SVVS), that checked whether a given
system complies to the standard.
\item POSIX (Portable Operating System Interface) is a standardization effort
of the IEEE organization (Institute of Electrical and Electronics Engineers).
\item SUSv4 is a common standard of The Open Group, IEEE (Std. 1003.1, 2008
Edition) and ISO (ISO/IEC 9945-2008).
\item To certify a given system for the Single Unix Specification, it is necessary
to pass a series of tests (on given architecture, e.g. 64-bit x86).
The results of the tests are then evaluated. The tests themselves are unified into
so called \emph{test suites}, which are sets of automatic tests that go through
the system and verify if it implements the interfaces specified in the standard.
For example, for SUSv3 there are 10 such test suites.
\item The interfaces specified by the POSIX.1-2008 standard are divided into 4
basic groups: XSH (System Interfaces), XCU (Shell and Utilities), XBD
(Base definitions). W.r.t. number of interfaces, the biggest of them is XSH which
contains more than 1000 interfaces.
\item The interface groups of POSIX together with the Xcurses group, are part
of the Single Unix Specification (however not part of POSIX base in the IEEE Std
1003.1-2001 standard) which includes 1742 interfaces in total, which form the Single Unix
Specification (2003). The SUS interface tables are here:
\url{http://www.unix.org/version3/inttables.pdf}
\item Commercial UNIXes largely follow the Single UNIX Specification, compliance
to this standard is the condition to use the UNIX trademark
(the UNIX 98 brand corresponds to SUSv2, UNIX 03 corresponds to SUSv3, SUSv4 is
UNIX V7 - do not mix it up with historical V7 UNIX). It is built on the POSIX
base.
\item We are going to follow SUSv4 for APIs in this lecture. The data structure
definitions and algorithms used by the kernel will be mostly based on
System~V Rel.~4 to keep things simple.
\item On Solaris there is an extensive \texttt{standards(5)} manual page, where
lots of information about standards can be found in one place.
Individual commands compliant to the standard are moreover placed
into special directories, e.g. the \texttt{tr} program is located in
\texttt{/usr/xpg4/bin/} and \texttt{/usr/xpg6/bin/} directories, in each there
is a version of the program compliant to the respective standard.
The options and behavior specified by the standard can be then relied upon e.g.
when writing shell scripts.
\item Also on Solaris, look into the
ifdef([[[NOSPELLCHECK]]], [[[
\texttt{/usr/inc{}lude/sys/fea\-ture\-\_tests.h}]]]) header file.
\end{itemize}

%%%%%%%%%%%%%%%%%%%%%%%%%%%%%%%%%%%%%%%%%%%%%%%%%%%%%%%%%%%%%%%%%%%%%%%%%

\begin{slide}
\centerslidestrue
\sltitle{The End.}
\end{slide}

\end{document}
