\AtBeginDocument[seminar]{%
  \pdfhorigin=1sp
  \pdfvorigin=1sp
  \paperwidth=297truemm
  \paperheight=210truemm
}

\documentclass[article]{seminar}

\usepackage{nth}

\usepackage[bookmarks,breaklinks=true,%
	pdftex,%
	pagebackref=true,%
	]{hyperref}

\usepackage{url}

%%% Macros %%%

% (janp)
\chardef\clqq=254  \sfcode254=0 \lccode254=0
\chardef\crqq=255  \sfcode255=0 \lccode255=0
%\DeclareRobustCommand\uv[1]{{\leavevmode\clqq#1\crqq}}
\DeclareRobustCommand\uv[1]{{\leavevmode{},,#1''}}
% (janp) end

\renewcommand{\slidelabel}{}
\setlength{\textwidth}{0.9\textwidth}

\newcommand{\sltitle}[1]{{\centering\textbf{\Large #1}
    \vskip 2em plus 0pt minus 2em\par}} % Slide title

\newcommand{\emsl}[1]{\textbf{#1}} % Emphasizing in slides

\newcommand{\funml}[1] { % Multi-line function prototype
    \begin{minipage}{\slidewidth}
    \vspace{-1ex}\texttt{\begin{tabbing}#1\end{tabbing}}
    \end{minipage}}

\centerslidesfalse

\begin{document}

\begin{slide}
\sltitle{Contents}
\end{slide}

\begin{itemize}
\item This lecture is mostly about Unix principles and Unix programming in the~C
language.
\item \emsl{The lecture is mostly about system calls, i.e. an interface between a
user space and system kernel.}
\item For the API, we will follow the \emph{Single UNIX Specification,
version~4} (SUSv4). Systems that submit to the Open Group for certification and
pass conformance tests are termed to be compliant with the UNIX standard
UNIX~V7.  Some versions of AIX, HP-UX and macOS on selected architectures
are compliant with the previous version SUSv3
(\url{http://www.opengroup.org/openbrand/register/xy.htm}).
\item The specific source code examples linked from this material are usually
tested on Solaris, macOS and Linux.
\end{itemize}

%%%%%
\pdfbookmark[0]{intro, programming utilities}{intro}

\begin{slide}
\sltitle{Contents}
\end{slide}

\pdfbookmark[1]{Current UNIX and Unix-like Systems}{currentunix}

\begin{slide}
\sltitle{Proprietary UNIX and Unix-like Systems}

\begin{itemize}
\item Sun Microsystems, now Oracle: \emsl{SunOS} (defunct), \emsl{Solaris}
\item Apple: \emsl{macOS} (formerly Mac OS X, Mac OS)
\item SGI: \emsl{IRIX} (in maintenance mode)
\item IBM: \emsl{AIX}
\item HP: \emsl{HP-UX}, \emsl{Tru64 UNIX} (defunct, formerly by Compaq)
\item SCO: \emsl{SCO Unix} (discontinued)
\item BSD/OS: \emsl{BSDi} (discontinued)
\item Xinuos (formerly Novell): \emsl{UNIXware}
\end{itemize}
\end{slide}

\begin{slide}
\sltitle{Open source Unix-like Systems}

\begin{itemize}
\item rather extensive number of \emsl{Linux} distributions
\item \emsl{FreeBSD}
\item \emsl{NetBSD}
\item \emsl{OpenBSD}
\item \emsl{DragonflyBSD}
\begin{itemize}
\item all BSD variants have roots in the 4.3BSD-Lite source code
\end{itemize}
\item \emsl{Minix}, micro-kernel based
\item \emsl{Illumos}, based on Solaris
\end{itemize}
\end{slide}

\begin{itemize}
\item Note that \emsl{Linux is a kernel}, not the whole system.  In contrast to
FreeBSD for example, which covers both the kernel and the userland.  It is
better to say a ``Linux distribution'' if you discuss a whole system that is
built around the Linux kernel.
\item FreeBSD and NetBSD forked from 386BSD (now defunct) in 1993, OpenBSD
forked from NetBSD in 1995, and DragonflyBSD forked from FreeBSD in 2003.
386BSD itself was based on 4.3BSD-Lite.  However, the history is much more
complicated, as usual.
\item Presently, the ``UNIX'' trademark can be only used by systems that passed
conformance tests defined in the Single UNIX Specification (SUS).
\item From those systems listed above, only macOS, AIX, and HP-UX are
UNIX~03 compliant (\url{http://www.opengroup.org/openbrand/register/}).  Other
non-certified systems, are often described as ``Unix-like'', even when in many
cases they closely follow the standard.  However, the word ``Unix'' is often used
for systems from either group.
\item The above list is a tiny fraction of the whole Unix world.  Every
proprietary Unix variant likely came from either UNIX~V or BSD, and added its
own features.  This resulted in quite a few standards as well, see page
\pageref{UNIXSTANDARDS}.  In the end vendors agreed upon a small set of those.
\item If you are interested in a detailed and up-to-date Unix system version
history, go check \url{https://www.levenez.com/unix/}.
\end{itemize}

%%%%%

\pdfbookmark[1]{UNIX standards}{unixstd}

\begin{slide}
\sltitle{UNIX standards}
\begin{itemize}
\renewcommand{\baselinestretch}{0.8}
\item \emsl{SVID} (System~V Interface Definition)
\item \emsl{POSIX} (Portable Operating System based on UNIX)
\item \emsl{XPG} (X/Open Portability Guide)
\item \emsl{Single UNIX Specification}
\end{itemize}
\end{slide}

\label{UNIXSTANDARDS}

\begin{itemize}
\item The very basic information is that the area of UNIX standards is very
complex and incomprehensible on a first sight.
\item AT\&T allowed the producers to call its own commercial UNIX variant
``System V'' only if it complied to the SVID standard conditions. AT\&T also
published \emph{System~V Verification Suite} (SVVS), that checked whether a given
system complies to the standard.
\end{itemize}

\end{document}
